\documentclass[twocolumn]{aastex631}
\begin{document}
\title{Investigation of Non-linear Oscillatory Modes in Neutron Star Accretion Disks and their Impact on Quasi-Periodic Oscillations}

\author{Your Name}
\affiliation{Your Institution}

\begin{abstract}
This paper explores the complex dynamics of neutron star (NS) accretion disks, focusing on the impact of non-linear oscillatory modes on the generation of quasi-periodic oscillations (QPOs). We propose a comprehensive modeling approach that incorporates multi-dimensional simulations, magnetic fields, and various dynamical instabilities in order to elucidate the physics underlying the observed X-ray variability in neutron star systems. The potential implications for gravitational wave observations and equation of state constraints are also discussed.
\end{abstract}

\section{Introduction}
Neutron stars (NSs) are remarkable astronomical objects that exhibit complex X-ray variability patterns, which are often analyzed through quasi-periodic oscillations (QPOs). These oscillations can provide insight into the extreme environments present in NS systems, offering valuable information regarding their internal structure and the equation of state (EoS) of dense matter. The observed high amplitudes of NS QPOs, especially in the kilohertz range, remain a topic of active research. Previous studies have suggested that the dynamics of accretion disks play a crucial role in shaping these signals.

The primary objective of this paper is to investigate the role of model complexity in understanding QPOs. By exploring non-linear oscillatory modes, we aim to shed light on their influence on energy transfer mechanisms, luminosity modulation, and the overall variability of the X-ray emissions from accreting neutron stars.

\section{Methodology}
To study the effects of non-linear oscillatory modes in neutron star accretion disks, we will include the following key components in our model:

\subsection{Non-linear Oscillation Models}
We will formulate equations governing non-linear oscillations of the accretion disk using the Navier-Stokes equations along with appropriate boundary conditions. The evolution of density and velocity fields will be modeled to analyze the interplay between disk dynamics and oscillatory behavior.

\subsection{Multi-Dimensional Simulations}
We will employ a multi-dimensional simulation framework (e.g., using Adaptive Mesh Refinement methods) to numerically solve the governing equations of the disk. This approach allows for high-resolution modeling of turbulent flows and the effects of instabilities within the accretion disk.

\subsection{Inclusion of Magnetic Fields}
The magnetic fields will be incorporated into our simulations based on magnetohydrodynamic (MHD) principles. We will explore how magnetic interactions modify disk structure, stability, and energy transfer processes.

\subsection{Dynamical Instabilities}
A thorough analysis of various instabilities that may arise in the accretion disk, such as the magneto-rotational instability (MRI) and the Kelvin-Helmholtz instability, will be conducted. We will investigate how these instabilities affect the oscillatory modes and impact the fluxes that lead to enhanced QPOs.

\section{Anticipated Results}
We expect that our investigation will reveal significant correlations between non-linear oscillatory modes and the characteristics of QPOs observed in NS systems. The results could indicate that:
\begin{itemize}
    \item Non-linear effects lead to increased amplitude and frequency variability in the X-ray emission.
    \item Magnetic field interactions suppress or enhance certain oscillatory modes, influencing QPO patterns.
    \item Dynamical instabilities create non-standard luminosity fluctuations, contributing to QPO complexity.
\end{itemize}

\section{Conclusion}
This study will provide a comprehensive understanding of how model complexity, particularly through non-linear oscillatory modes and magnetic field effects, influences the observed QPOs in neutron star accretion disks. The implications for future gravitational wave observations and EoS constraints based on QPOs will also be discussed.

\section*{Acknowledgments}
We wish to thank XXX for their guidance and support in this research. This work is supported by XXX and was performed using resources provided by XXX.

\bigskip
\bibliography{yourbibfile}
\end{document}